\chapter*{Introduction}
\addcontentsline{toc}{chapter}{Introduction}

OLAP \cite{berson1997} or data cubes are established as one of the standard tools for analysing aggregated data and are an essential part of any Bussiness Intelligence solution. Any transactional data, that relates to multiple contexts (e.g. multinational fast food chain's sales can be divided with respect to regions, marketing campaigns etc.) can be transformed into their multidimensional representation in a form of OLAP cube where the contexts potentially relevant for analysis make up the dimensions of the cube and be examined in one of many available standard BI tools (such as Microsoft Excel, Power BI etc.), that facilitate aggregation of the values and manipulation with the cube in order to gain insights into the data.

Data mining can function as an alternative or complement \cite{Chudan2015} to OLAP analysis. The term data mining refers to extracting or \textit{mining} knowledge from large amounts of data (Hand et al.,2001 TODO cite), usually relational databases where the data is on the most granular level. The purpose of data mining is either \textit{descriptive}, where it is aimed to find and describe regularities and patterns in the data, or \textit{predictive}, where the goal is to infere new information hidden in the data, that can be used to make predictions. Association rules are on of the data mining methods and can be used for descriptive and predictive analysis. The rules can be mined by different algorithms, namely Apriori algorithm \cite{Agrawal1993} and the ASSOC procedure of the GUHA method.\cite{hajek1966} They both serve to mine association rules in tabular data.

Data representation is not only limited to tables. In the environment of World Wide Web, data that is intented to be published to the public, is often modelled according to the Resource Description Framework (RDF) model. Individual records in the RDF model take a form of \textit{triples}, that represent binary relationships between the described entities. Sets of the relationships between entities and concepts that those entities can represent are call \textit{vocabularies} or \textit{ontologies} and are published as RDF as well. The relationships can connect entities from different data sources, making the data \textit{linked}. The term \textit{Linked Open data} (LOD) has become established for RDF data, which adheres to the use of standard formats, technologies and interconnection principles that facilitate and expand the possibilities of working with this data. 

An algorithm, that was designed specifically for the purposes of mining rules similar to association rules from the RDF data in the Web environment, is called \textit{\textbf{A}ssociation Rule \textbf{M}ining under \textbf{I}ncomplete \textbf{E}vidence}, shorty AMIE. \cite{Galarraga2013} Since the release of the original version of the algorithm, its original authors have published its two extensions, called AMIE+ \cite{Galarraga2015} and AMIE 3 \cite{Galarraga2020}, respectively. Another extension that builds on the AMIE + version, called RDFRules \cite{Zeman2020}, comes with its reference implementation in the form of a robust framework, which focuses not only on the rule mining itself but also on the possible preprocessing of input data and processing the generated rules.

Public-sector organizations and governmental bodies, that manage a large amount of data including various registers and demographic and economic statistics are often incentivised to publish some of their data for the benefit of their citizens. Due to the data privacy restrictions \cite{gdpr162_2020}, the organisations often resort to publish their data in an aggregated form. Some of the organizations, such as the European Union through its \textit{OpenBudgets.eu}\footnote{\href{https://ec.europa.eu/digital-single-market/en/content/openbudgetseu}{https://ec.europa.eu/digital-single-market/en/content/openbudgetseu}} platform or the Czech Social Security Administration\footnote{\href{https://data.cssz.cz/}{https://data.cssz.cz/}}, acknowledge the advantages of publishing their data in a universal and standard way for the consumers and do not hesitate to devote their resources to publishing their data as LOD. The Data Cube vocabulary \cite{dcv2014} is used to write aggregated data in the form of a data cube in the RDF model.

The interlinked nature of LOD encourages the published data cubes to be enriched with additional information available from other sources published as RDF as well. The connection would be made through those data cube dimension values that simultaneously occur as objects in large cross-domain LOD data set known as \textit{knowledge graphs}. Those can be municipalities, regions, organizations, products, etc. This new information in the form of binary relationships contained in these knowledge graphs can be used to mine association rules over the aggregated data, which can lead to finding relationships that cannot be found in the cubes themselves. Mining of association rules over RDF data and at the same time in their aggregated form is a not yet explored area and achieving generation of meaningful interpretable rules that bring new knowledge is a not yet solved problem.

The goals of this work are to:

\begin{enumerate}
    \item explore the possibilities of enriching RDF data cube structured by the Data Cube vocabulary with the data from general knowledge graphs and of mining such data by the AMIE algorithm or its derivatives,
    \item carry out an experiment that would demonstrate that such approach is possible and capable of yielding new interesting insights on the aggregated data. 
\end{enumerate}

The RDFRules framework is one of the tools used for performing the experiment. The aggregated data examined in the experiment are Czech pension statistics published by Czech Social Security Administration and Czech demography statistics of the Czech statistical office\footnote{\href{https://www.czso.cz/csu/czso/home}{https://www.czso.cz/csu/czso/home}}. The triples associated with the dimension values of the data cube are extracted from Wikidata\footnote{\href{https://www.wikidata.org/}{https://www.wikidata.org/}} and YAGO\footnote{\href{https://yago-knowledge.org/}{https://yago-knowledge.org/}} knowledge graphs.

This paper is organised as follows. Section \ref{linkeddata} provides an overview of Linked Open Data principles, RDF data model and its serialization formats, and the data sets used in the experiment. Section \ref{datacubes} introduces the basic concepts of OLAP cubes and describes the vocabularies used for representing the data cubes in RDF. Section \ref{assocrules} explains the basics of the association rule mining. Section \ref{amieandits} describes the AMIE algorithm, the improvements of its extension AMIE+ and presents the improvements of RDFRules algorithm and the techniques suggested to be used along with it. Section \ref{rdfrules} describes the reference implementation of the RDFRules. Section \ref{combining} elaborates the possibilities and implications of mining rules over RDF data cubes with combination of triples from knowledge graphs. Section \ref{experiment} describes thoroughly the performed experiment. Section \ref{discussion} elaborates the results of the executed mining tasks and encountered problems and the conclusion summarizes the contribution of the work and gives suggestions for further activities.

% i vysvětlení, co bude v dalších kapitolách jako „preliminaries“ – linked data, data cubes a dolování pravidel

% co je v kapitole 4 (ze tam jsou algoritmy které vysvetlují background frameworku RDFRules), kapitola 5 je o frameworku, kapitola 6, kapitola 7, kapitola 8