\chapter{Leveraging a Combination of OLAP Cubes and Knowledge Graphs}

% TODO popřemýšlet o jiném názvu

% TODO co je v zadání od Svátka

% k hodnotám dimenzí datové kostky se můžou z verejných zdrojů napojit nové vlastnosti, RDF dovoluje snadnou integraci, RDFRules rozpoznává equals

\section{Mining from RDF representation of Data Cube}

% TODO že se dají kostky reprezentovat slovnikem data cube vocabulary viz ...

% TODO souměřitelnost -> způsoby řešení, rozsekání, clustering, viz Chudán

% TODO problémy s mírami 1) nesouměřitelnost 2) více různé datasety stejná míra !!! (ukázat priklady)

% TODO že se musí diskrezitovat, jaké jsou způsoby diskretizace, jaký to má vliv na measures: lift, HC, support, když se to disktretizuje určitým způsobem, toto rozvést víc a uvést klidně i nějaký příklad (chtel VS) 

% TODO zmínit Chudána

% TODO chudán 4.1 Combining OLAP and association rules

\section{Appending RDF Data to the Data Cubes}

% TODO že se to napojuje k hodnotám dimenzí, které jsou v té kostce (že se dá třeba připojit něco k času, nebo k dimenzi zákazník, oblast)

% TODO moje výstupy