\chapter{Experiment}

% TODO nějaký úvod, že ukazuju, jak je možné kombinovat OLAP a KG, co jsem použil za technologie, proč to dělám REPL

This section describes an experiment of mining associtation rules from RDF data compiled of statistical data structured by the Data Cube Vocabulary and facts pulled from the Wikidata data set that was performed as an practical part of this work. The statistical data come from two sources. The first one is the Czech Social Security Administration and the second one is the Czech Statistical Office. Analysis was performed using the Scala API of the reference implementation of the RDFRules algorithm. The following sectins describe how the available data had to be preprocced to give reasonable results in combination with KG data. The preprocessing was perfomed partly by the implementation's API itself, partly by performing SPARQL queries over the data. The described method can be taken inspiration from when performing similar analysis ie. association rule mining task over the multidimensional data merged with loosely structured graph data. 

\section{Czech Social Security Administration}

Czech Social Security Administration (CSSA) is a czech public administration organtisation responsible for collecting social security premiums and contributions to the state employment policy. Since 2015 the organization publishes its statistical yearbook datasets and other (vocabularies, code lists and datasets containing data concerning the internal operation of the organization) in the form LOD and became of the first czech public institutions to do so. The yearbook statistical data sets are modelled using Data Cube Vocabulary. Their dimension values are represented by the SKOS vocabulary. The organization has published 73 datasets so far. All these datasets are downloadable as dumps\footnote{\href{http://data.cssz.cz/web/otevrena-data/katalog-otevrenych-dat}{http://data.cssz.cz/web/otevrena-data/katalog-otevrenych-dat}} or accesible througth a SPARQL endpoint\footnote{\href{http://data.cssz.cz/web/otevrena-data/sparql-query-editor/}{http://data.cssz.cz/web/otevrena-data/sparql-query-editor/}}. The CSSA's URI are dereferenceable.

% project vše
% že tam jsou i CSV data a pro každý data set je stránka s popisem

The largest of the data cubes published is \verb|cssa-d:duchodci-v-cr-krajich-okresech|\footnote{https://data.cssz.cz/resource/dataset/duchodci-v-cr-krajich-okresech}. From now on it will be denoted as CSSA1. It contains \numprint{368118} observations structured spread over four dimensions: reference area\footnote{https://data.cssz.cz/ontology/dimension/refArea}, reference period\footnote{https://data.cssz.cz/ontology/dimension/refPeriod}, sex\footnote{https://data.cssz.cz/ontology/dimension/pohlavi} and pension kind\footnote{https://data.cssz.cz/ontology/dimension/druh-duchodu}. Observations are assigned three measures: the average amount of pension\footnote{https://data.cssz.cz/ontology/measure/prumerna-vyse-duchodu-v-kc}, the average age\footnote{https://data.cssz.cz/ontology/measure/prumerny-vek} and the number of persons\footnote{https://data.cssz.cz/ontology/measure/pocet-duchodcu}. Each observation is assigned only one measure.

\begin{figure}[h]
\begin{lstlisting}[language = turtle, caption={Example of an observation from CSSA1}, label={cssa1example},captionpos=b escapeinside={(*@}{@*)}]
@prefix qb: <http://purl.org/linked-data/cube#> .
@prefix cssa-om: <https://data.cssz.cz/ontology/measure/> .
@prefix cssa-d: <https://data.cssz.cz/resource/dataset/> .
@prefix cssa-od: <https://data.cssz.cz/ontology/dimension/> .

<https://data.cssz.cz/resource/observation/duchodci-v-cr-krajich-okresech/2017-12-31/prumerna-vyse-duchodu-v-kc/pk_srnvm/vc.35/m>
    a qb:Observation ;
    qb:dataSet cssa-d:duchodci-v-cr-krajich-okresech .
    qb:measureType cssa-om:prumerna-vyse-duchodu-v-kc ;
    cssa-od:druh-duchodu <https://data.cssz.cz/resource/pension-kind/PK_SRNVM_2010> ;
    cssa-od:pohlavi <https://data.cssz.cz/ontology/sdmx/code/sex-M> ;
    cssa-od:refArea <https://data.cssz.cz/resource/ruian/vusc/35>;
    cssa-od:refPeriod <https://data.cssz.cz/resource/reference.data.gov.uk/id/gregorian-day/2017-12-31>;
    cssa-om:prumerna-vyse-duchodu-v-kc 6622.0;
\end{lstlisting}
\end{figure}

In this particular data set there are \numprint{102} distinct values of the dimension of reference area: 14 regions (NUTS 3 administrative units, czech translation in singular nominative is \textit{kraj}) including Prague, 77 districts (\textit{okres}) also including Prague, 10 Prague districts (\textit{správní obvod}) and a value representing the state in total. Each entity representing a reference area is assigned an unique numerical identifier which corresponds to this area's identifier in the official Registry of Territorial Identification, Addresses and Real Estate (RTIAR) runned by the State Administration of Land Surveying and Cadastre (SALSC). RTIAR codes are reference codes by law, so it is obligatory for CSSA to use them and have them correct.

% SPARQL dotaz, ktery ukaze ty hodnoty ?

\begin{figure}[h]
\begin{lstlisting}[language = turtle, caption={Dereferenced proxy entity of the South Bohemian Region}, label={sbrexample},captionpos=b, escapeinside={(*@}{@*)}]
@prefix skos:  <http://www.w3.org/2004/02/skos/core#> .

<https://data.cssz.cz/resource/ruian/vusc/35>
    a <https://data.cssz.cz/ontology/ruian/Vusc> , skos:Concept ;
    <http://www.w3.org/2002/07/owl#sameAs> <https://linked.cuzk.cz/resource/ruian/vusc/35> ;
    skos:inScheme  <https://data.cssz.cz/resource/ruian/ConceptScheme> ;
    skos:notation "VC.35" ;
    skos:prefLabel "(*@Jihočeský kraj@*)" .
\end{lstlisting}
\end{figure}

There are official URIs of this registry but CSSA datasets do not used them directly. The entities for the dimension of reference area and other dimensions in the dataset CSSA1 and all other data sets of CSSA with the dimension of reference area work as so-called \textit{proxy entities}. This means that instead of using the original code list item URIs directly as objects in the RDF triples,  it uses their equivalents defined in the internal code lists. These equivalents are connected to the original URI by the \verb|owl:sameAs| statement. These proxies can then contain data specific to CSSA e.g. labels. Another advantage of this is, that these URIs are dereferenceable to the CSSA domain and their versioning is under the control of CSSA and they can be easily redirect to a different equivalent code list. Previously the proxy entities of the reference area were directed to the unofficial transformation of the Opendata.cz initiative\footnote{\href{https://linked.opendata.cz/dataset/cz-ruian}{https://linked.opendata.cz/dataset/cz-ruian}}.

Another dimension whose values work as proxy entities is the dimension of reference period. This dimension divides the observations into one year intervals. This applies to all other CSSA data cubes containg the reference period dimension. The data set vary in the overall covered period. The last covered year of all data sets is the year 2019. The entities link to the \verb|data.gov.uk| Time Intervals\footnote{\href{http://old.datahub.io/dataset/data-gov-uk-time-intervals}{http://old.datahub.io/dataset/data-gov-uk-time-intervals}} OWL ontology. Usage of this entities is, however, not unified. In some data sets intervals are assigned an entity representing a year, in other they are assigned an entity representing the last day of the corresponding year. All of them are, however, representing a period of a whole year.

\begin{figure}[h]
\begin{lstlisting}[language = turtle, caption={Dereferenced proxy entity of the year 2017}, label={gd2017example},captionpos=b, escapeinside={(*@}{@*)}]
@prefix skos:  <http://www.w3.org/2004/02/skos/core#> .

<https://data.cssz.cz/resource/reference.data.gov.uk/id/gregorian-year/2017>
    a skos:Concept ;
    <http://www.w3.org/2002/07/owl#sameAs> <http://reference.data.gov.uk/id/gregorian-year/2017> ;
    skos:inScheme <https://data.cssz.cz/ontology/years/YearsScheme> ;
    skos:notation "2017" ;
    skos:prefLabel "2017" .
\end{lstlisting}
\end{figure}

The CSSA1 dataset uses two distinct schemes for the pension kinds, because in 2010, the official categorization of pensions was changed in the czech legislation. Only the observations assigned to year 2008 are divided according to the old pension scheme. All other year's observations correnspond to the new pension scheme. So one can not simply multiply the numbers of distinct values for each dimension and the number of measures to get the total number of observations for this particular cube. The URIs of both pension kind schemes are suffixed either by \verb|_2008| (31 of them) for the old scheme or \verb|_2010| (37 of them) for the new scheme. Not all entities correspond to a particular kind of pension. Some of them represent an aggregation over related pension kinds or simply an aggregation over all of the pension kinds.

\begin{figure}[h]
\begin{lstlisting}[language = turtle, caption={Dereferenced pension kind}, label={pkexample},captionpos=b, escapeinside={(*@}{@*)}]
@prefix skos:  <http://www.w3.org/2004/02/skos/core#> .

<https://data.cssz.cz/resource/pension-kind/PK_SRNVM_2010>
    a <https://data.cssz.cz/ontology/pension-kinds/PensionKind> , skos:Concept ;
    skos:altLabel "SRNVM"@cs ;
    skos:exactMatch <https://data.cssz.cz/resource/pension-kind/PK_SRNVM> ;
    skos:inScheme <https://data.cssz.cz/ontology/pension-kinds/PensionKindScheme_2010> ;
    skos:notation "PK_SRNVM" ;
    skos:prefLabel "(*@Starobní důchod SRN vyplácený v souběhu s vdoveckým důchodem@*)"@cs .
\end{lstlisting}
\end{figure}

The dimension of sex consists of three distinct values: dimension of male pension, dimension of female pensions and its total. The values are proxy entities linking to the SDMX representations of sexes\footnote{\href{http://purl.org/linked-data/sdmx/2009/code}{http://purl.org/linked-data/sdmx/2009/code}}

\section{Czech Statistical Office}

The Czech Statistical Office (CZSO) is the main public organization responsible for collecting and analyzing statistical data in the Czech Republic. This organization is for example responsible for the state's census. Data about demography, economics, education, health care etc. are made availabe on the organization's website\footnote{\href{https://vdb.czso.cz/vdbvo2/faces/en/index.jsf?page=uziv-dotaz}{https://vdb.czso.cz/vdbvo2/faces/en/index.jsf?page=uziv-dotaz}} in a form of interactive spreadsheet builder. Thanks to the Opendata.cz initiative this data sets are made available as LOD modelled by the Daca Cube Vocabulary in the initiative's catalogue. The data is also hosted as a SPARQL endpoint\footnote{\href{http://linked.opendata.cz/sparql}{http://linked.opendata.cz/sparql}}. 8 of these data sets have a dimension of reference period. Each data set’s dump file can be downloaded from the catalouge\footnote{The download link URLs are, however, broken and return HTTP status code 404. To get the dump file, word \textit{dumps} has to be substituted with word \textit{soubor} (czech word for \textit{file}). For example, the dump file of the data set \textbf{czso-job-applicants} is available at \href{https://linked.opendata.cz/soubor/czso-job-applicants.trig}{https://linked.opendata.cz/soubor/czso-job-applicants.trig}}.

\begin{figure}[h]
\begin{lstlisting}[language = turtle, caption={Example of an observation from the CZSO data sets}, label={turtleexample},captionpos=b escapeinside={(*@}{@*)}]
@prefix qb: <http://purl.org/linked-data/cube#> .
@prefix czso: <http://data.czso.cz/ontology/> .

<http://data.czso.cz/resource/observation/job-applicants-and-unemployment-rate/CZ0513/2009-12-31/T> a qb:Observation ;
    czso:refArea <http://ruian.linked.opendata.cz/resource/okresy/3505> ;
    czso:refPeriod <http://reference.data.gov.uk/id/gregorian-day/2009-12-31> ;
    czso:sex sdmx-code:sex-T ;
    czso:neumisteniUchazeciOZamestnani 9692.0 ;
    czso:dosazitelniNeumisteniUchazeciOZamestnani 9528.0 ;
    czso:podilNezamestnanych 7.92 ;
    czso:pocetVolnychMist 569.0 ;
    qb:dataSet <http://data.czso.cz/resource/dataset/job-applicants-and-unemployment-rate> .
\end{lstlisting}
\end{figure}

Observations in CZSO data cubes are assigned multiple measures. Their URIs are not dereferenceable. Their dimension value URIs do not work as proxy entities. The dimension of reference area uses entities of an above-mentioned initiative's unofficial RTIAR transformation\footnote{\href{https://linked.opendata.cz/dataset/cz-ruian}{https://linked.opendata.cz/dataset/cz-ruian}} with its own SPARQL endpoint\footnote{\href{https://ruian.linked.opendata.cz/sparql}{https://ruian.linked.opendata.cz/sparql}}. The measured values relate to regions and district. They do not contain observations related to the whole state. The proxy entities of the reference area dimension values for the CSSA data sets previously linked to this code list. 

For time intervals representation the CZSO data cubes also use the the \verb|data.gov.uk| Time Intervals OWL ontology. They only do so directly unlike the CSSA data cubes. The data cubes vary their time span. The earliest recorded values are for the year 2005. The latest values are for the year 2013. There are two data sets that contain values for both the earliest and latest year mentioned meaning they cover a period of 9 years: \textbf{czso-deaths-by-selected-causes-of-death}\footnote{\href{https://linked.opendata.cz/dataset/czso-deaths-by-selected-causes-of-death}{https://linked.opendata.cz/dataset/czso-deaths-by-selected-causes-of-death}} and \textbf{czso-job-applicants-and-unemployment-rate}\footnote{\href{https://linked.opendata.cz/dataset/czso-job-applicants-and-unemployment-rate}{https://linked.opendata.cz/dataset/czso-job-applicants-and-unemployment-rate}}

Just as with the CSSA data cubes, some of the CZSO data cubes contain the dimension of sex consisting of three distinct values: dimension of male pension, dimension of female pensions and its total. The values used are the SDMX representations of sexes themselves.

\section{Wikidata}

Wikidata data set contains data about political representation of countries, their administrative areas and municipalities. For the purposes of this experiment, such data concerning the Czech Republic was extracted from the data set. In the Czech Republic, regions and municipalities\footnote{Not districts though.} are being assigned a government that emerges from elections. In Wikidata data set, there exist records of who was or still is head of this local government, including the head of the state government. Records of these \textit{head of government} roles are given a time period of validity of this role by stating the date of this role's start and optionally the end of this role when it is not a current area government head anymore. For the persons who hold or held the office, the affiliation to a political party is stated in the data also with the start and end date of this affiliation. The entities of the political parties are assigned their political alignment (left, center, far-right etc.). In the sample of Wikidata data set's content below it is stated that since 2008 till 2016 the head of the government of the South Moravian Region was Michal Hašek who since 1998 is a member of the Czech Social Democratic Party, which has the centre-left political alignment.

\begin{figure}[h]
\begin{lstlisting}[language = turtle, caption={Description of XXX (Source: author)}, label={wdhasekexample},captionpos=b, escapeinside={(*@}{@*)}]
@prefix wd: <http://www.wikidata.org/entity/> .
@prefix p: <http://www.wikidata.org/prop/> .

wd:Q192697 rdfs:label "South Moravian Region"@en ;
    p:P6 [ p:P6 wd:Q6835752 ; p:P580 "2008-11-21T00:00:00Z" ; p:P582 "2016-11-16T00:00:00Z" ] .

wd:Q6835752 rdfs:label (*@"Michal Hašek"@*)@en ;
    p:P102 [ p:P102  wd:Q341148 ; p:P580 "1998-01-01T00:00:00Z" ] .

wd:Q341148 rdfs:label "Czech Social Democratic Party"@en ;
    p:P1387 [ p:P1387 wd:Q737014 ] .

wd:Q737014 rdfs:label "centre-left"@en .
\end{lstlisting}
\end{figure}

When adequately preprocced, this data can be utilized to find relations of statistical data described in the data cubes of CSSA and CZSO and the political cycle in the country. For example a rule can be found that states, that if in any year, the head of the Czech Republic's government was a member of a left-leaning policital party, the pension expenses for one-off allowance to pensions were above average. A query that extracts this data from the Wikidata's SPARQL endpoint is listed in \ref{wdextract}. This data, however, cannot be used for mining such rules yet. Measures in the data cubes are recorded on year to year bases. For the governmental roles and political affiliations it is only know the start date and end date. It is necessary to transform these triples into set of triples stating that a governmental role or the policital affiliation \textit{applies} for a certain year. The edge years are a bit tricky because the role or the membership was not valid for the whole year it started or ended. To facilitate the query and to generate more triples I decided to generate the triples for the edge years as well. The SPARQL query that constructs the \textit{appliesToRefPeriod} triples from the extracted data is listed in \ref{wdapplies}. The triples stating the start dates and end dates are no longer needed and do not have to be loaded into the RDFRules mining task.

% todo P582 a P580 nejsou už potřeba a proto se pak smažou ...

% jak jsem si to musel předzpracovat, vytvořit ty appliesTo

\section{Filtering the Observations}

The values for the city of Prague are duplicated. The city is assigned an entity both as a region and as a district. The administrative area of the Czech Republic's capital is given a special status by the Act No. 131/2000 Coll., on the Capital City of Prague and does not in fact fall into neighter of those categories. Nonetheless, Prague is assigned an identifier both as a district (\numprint{3100}) and as a region (\numprint{19}) in the RTIAR registry. When it comes to total population of the area (\numprint{1324227} as of 2020), it is comparable to other czech regions. The least populated is the Karlovy Vary Region with around \numprint{300000} inhabitants and the most populated: the Central Bohemian Region has around \numprint{1300000} inhabitants. Its surface area is on the other hand comparable with the districts. As the statistics about pensioners are certainly more correlated with the population rather than the surface area, the observation allocated to the dimension value of Prage as being district would be filtered out to maintain the commensurability along values measured for districts (see \ref{cssaSlicing}).

I also chose to filter out all observations regarding year 2008. For 2008 the pension kinds are structured according to a different scheme than any other year and it is hard to assume compatibility for the URIs that only differ in the year's suffix. 2008 scheme contains penkind kinds that 2010 does not end vice versa. It could be possible to just cut the cube so that the year's 2008 become one cube and the other years the other one, but a cube concerning only one reference period has not got much value. Both filters can be performed in a single SPARQL query. In the CSSA1 data set, discarting the Prague as a District entity removes \numprint{3609} observations. The year 2008 contained \numprint{28458} observations. \numprint{336330} observations are contained in the query's result making up 91,4\% of the unfiltered data set.

\begin{figure}[h]
\begin{lstlisting}[language = SPARQL, caption={SPARQL query to filter the CSSA1 data set (Source: author)}, label={sparqlexample},captionpos=b escapeinside={(*@}{@*)}]
PREFIX skos: <http://www.w3.org/2004/02/skos/core#>
PREFIX qb: <http://purl.org/linked-data/cube#>
PREFIX cssa-d: <https://data.cssz.cz/resource/dataset/>
PREFIX cssa-od: <https://data.cssz.cz/ontology/dimension/>
PREFIX cssa-rd: <https://data.cssz.cz/resource/ruian/okresy/>
PREFIX cssa-op: <https://data.cssz.cz/ontology/pension-kinds/>

CONSTRUCT {
    ?observation ?p ?o
} 
WHERE {
    GRAPH cssa-d:duchodci-v-cr-krajich-okresech {
        ?observation qb:dataSet cssa-d:duchodci-v-cr-krajich-okresech ;
            cssa-od:druh-duchodu ?druh ;
            ?p ?o .
        NOT EXISTS {
            ?observation cssa-od:refArea cssa-rd:3100 .
        }
     }
     GRAPH cssa-d:pomocne-ciselniky {
        ?druh skos:inScheme cssa-op:PensionKindScheme_2010 .
     }
}
\end{lstlisting}
\end{figure}

\section{Slicing the Cubes\label{cssaSlicing}}

It is in part aimed to mine rules, in which the head atom's predicate is one of the cube's measures. In order to ensure, that such rules can achieve a reasonable support, the numerical values at the position of object in the measure triples have to be discretized and replaced by intervals. Irrespective to a chosen discretization approach, it is inadmissible to discretize values belonging to different disproportionate contexts. For example, we cannot create intervals for the number of pensioners from values measured for both regions and districts together. A district is a lower administrative unit. It belongs to a lower level in the concept hierarchy and it is assumed that its numbers of pensioners are of a different order of magnitude than those for regions or for the whole state. Same applies for values of dimensions sex and pension kind of the described data set. The values of the reference period represent even time intervals so the commensurability is assumed.

One way to solve is to slice a preprocessed cube having disproportionate dimension values into a set of smaller subcubes, in which the dimension values belong to the same level of a concept hierarchy. Measured values can be then discretized into intervals in each subcube separately. Number of subcubes that the main cube has to be divided into depends on the number dimensions and the number of levels in each dimension's hierarchies. Also when the commensurability cannot be expected among dimension values on the same level of their hierarchy, it is a good idea to \textit{make a cut} for each dimension value. For example, the dataset \verb|cssa-d:vydaje-na-duchody-v-cr|\footnote{\href{https://data.cssz.cz/resource/dataset/vydaje-na-duchody-v-cr}{https://data.cssz.cz/resource/dataset/vydaje-na-duchody-v-cr}} capturing costs on pensions in the Czech Republic by year and kind of pension contains 10 distinct values of the dimension of pension kind (not considering the scheme used only for year 2008). The cube would have to be divided into 10 subcubes for each value of the pension kind dimension. The CSSA1 data set would have to be divided into 222 subcubes. It has 37 pension kinds. Dimension of sex has two hierarchy levels: each sex and total. The dimension of reference area is considered to have 3 hierarchy levels: State's total, regions and Prague districts combined with the regional districts since they are comparable in number of inhabitants.

$$
37 \times 2 \times 3 = 222\hspace{0.4em}subcubes
$$

The construction of a subcube from a \textit{master} cube can be performed by a SPARQL CONSTRUCT query. An example of such query is shown in the listing \ref{sparqlcutexample}. This query filters the triples of the CZSO data cube \textbf{czso-job-applicants-and-unemployment-rate} to create a smaller cube of statistics about job applicants and unemployment rate for districts by sex. Notice how the reference area values corresponding to districts are distinguished. After the reference area values are linked (see \ref{linking}) to their CSSA counterparts, the ontology provided with the CSSA data sets can be reused.

\begin{figure}[h]
\begin{lstlisting}[language = SPARQL, caption={SPARQL query to create a subcube (Source: author)}, label={sparqlcutexample},captionpos=b escapeinside={(*@}{@*)}]
PREFIX qb: <http://purl.org/linked-data/cube#>
PREFIX sdmx-c: <http://purl.org/linked-data/sdmx/2009/code#>
PREFIX czso: <http://data.czso.cz/ontology/>
PREFIX czso-rd: <http://data.czso.cz/resource/dataset/>
PREFIX cssa-rd: <https://data.cssz.cz/resource/dataset/>
PREFIX cssa-or: <https://data.cssz.cz/ontology/ruian/>
PREFIX owl: <http://www.w3.org/2002/07/owl#>
    
CONSTRUCT { ?observation ?p ?o } 
WHERE { 
    GRAPH czso-rd:job-applicants-and-unemployment-rate {
        ?observation qb:dataSet czso-rd:job-applicants-and-unemployment-rate ;
            ?p ?o ;
            czso:refArea ?refAreaCZSO .
        NOT EXISTS {
            ?observation czso:sex sdmx-c:sex-T .
        }             
     }
    ?refAreaCSSZ owl:sameAs   ?refAreaCZSO.
    GRAPH cssa-rd:pomocne-ciselniky { ?refAreaCSSZ a cssa-or:Okres }
}
\end{lstlisting}
\end{figure}

For every subcube a similar query has to be created and performed over the master cube. For a cube that has to be divided into a small number of subcubes it is plausible to write (and save it for the documentation and repeatability purposes) and perform these queries manually. But there are cubes for which this would involve an hours long work. At the same time, it is an trivial activity that can easily be automated. For this preproccesing step for the CSSA1 dataset, a Scala script was written that creates 222 distinct SPARQL queries that construct 222 subcubes, saves each query to a text file and also creates a shell script that triggers all queries and saves a result of each query to a distinct file in the turtle format. The script is listed in \ref{scalascript}. This, however, still requires writing such script for each preprocced data cube and solve the problem of a time consuming resolution of the queries.


% že jsem řezal SPARQLem a ne RDFRules API, do ní už šly nařezané kostky

% že se musel potom upravit triply qb:dataSet object

\section{Linking\label{linking}}

In order to find rules describing relations across multiple sources (meaning data cubes of CZSO, data cubes of CSSA and Wikidata triples) the entities either have to be assigned the same URIs or to be connected by the \verb|owl:sameAs| statements. The shared dimensions of the CSSA and CZSO data cubes are the reference area, reference period and sex. The dimension values URIs used for these dimensions differ not only institution from institution but also data cube from data cube from the same institution (In the CSSA data cubes, reference period is represented by an entity of a year and of the last day of the year as well). Linking of equivalent dimension values of the three dimensions is done by creating \verb|owl:sameAs| statements.

\subsection{Sex Dimension Values}

It was already mentioned that the CSSA data cubes use proxy entities linking to the SDMX representations of sexes, whereas the CZSO data cubes use these representations directly. So the linking statements are already provided with the CSSA code list file. To extract these very triples, the query listed below can be used. These triples can be than loaded into a mining task involving mining from data cubes contain the dimension of sex instead of loading the whole code list file.

\begin{figure}[h]
\begin{lstlisting}[language = SPARQL, caption={Linking the sex dimension values}, label={sexlinkextract},captionpos=b, escapeinside={(*@}{@*)}]
PREFIX owl: <http://www.w3.org/2002/07/owl#>

CONSTRUCT { ?cssaSex owl:sameAs ?sdmxSex }
WHERE {
    GRAPH <https://data.cssz.cz/resource/dataset/pomocne-ciselniky> {
        ?cssaSex a <https://data.cssz.cz/ontology/sdmx/code/Sex> ; owl:sameAs ?sdmxSex
    }
}
\end{lstlisting}
\end{figure}

\subsection{Reference Periods}

In CSSA data cubes, values from two concept schemes are used for representing the year intervals: a years scheme and a days scheme. The entities in the schemes are proxy entitities linking to the \verb|data.gov.uk| Time Intervals ontology. CZSO data cubes use the ontology's day scheme concepts directly. That means that every year is represented by three distinct URIs in the data cubes so two \verb|owl:sameAs| statements are required for each year. A query was written that generates theses statements for every year entity in the CSSA code list:

\begin{figure}[h]
\begin{lstlisting}[language = SPARQL, caption={Linking the reference periods}, label={sexlinkextract},captionpos=b, escapeinside={(*@}{@*)}]
PREFIX owl: <http://www.w3.org/2002/07/owl#>
PREFIX skos: <http://www.w3.org/2004/02/skos/core#>
    
CONSTRUCT {
    ?cssaYear owl:sameAs  ?cssaDay .
    ?cssaYear owl:sameAs ?dataGovDay .
    ?cssaDay owl:sameAs ?dataGovDay .
}  
WHERE {
    GRAPH <https://data.cssz.cz/resource/dataset/pomocne-ciselniky> {
        ?cssaYear skos:inScheme <https://data.cssz.cz/ontology/years/YearsScheme>
        BIND (REPLACE(str(?cssaYear),".*(\\d{4})","$1") as ?cssaYearValue)
        ?cssaDay skos:inScheme <https://data.cssz.cz/ontology/days/DaysScheme>
        FILTER (REGEX(str(?cssaDay), ".*day.*12-31"))
        BIND (REPLACE(str(?cssaDay),".*(\\d{4})-12-31","$1") as ?cssaDayValue)
        ?cssaDay owl:sameAs ?dataGovDay
        FILTER (?cssaYearValue = ?cssaDayValue)
    }
}
\end{lstlisting}
\end{figure}

\subsection{Reference Areas}

The proxy entities of the reference area in CSSA data set used to link to the same entities that are used by the CZSO data sets. This linking is no longer present in the CSSA's code list but can be retrieved from the Opendata.cz's SPARQL endpoint. The query listed below returns 114 linking triples for all districts (including the Prague district entity), all regions (including Prague), Prague districts and the entity of the whole state. The linking with the Wikidata's entities had to be performed manually.

\begin{figure}[h]
\begin{lstlisting}[language = SPARQL, caption={Linking the reference periods}, label={refarealinking},captionpos=b, escapeinside={(*@}{@*)}]
PREFIX owl: <http://www.w3.org/2002/07/owl#>

CONSTRUCT {
    ?cssaArea owl:sameAs ?odArea
}
WHERE {
    ?cssaArea a ?class ; owl:sameAs ?odArea .
    FILTER (?class IN (
        <https://data.cssz.cz/ontology/ruian/Okres>,
        <https://data.cssz.cz/ontology/ruian/Vusc>,
        <https://data.cssz.cz/ontology/ruian/SpravniObvod>,
        <https://data.cssz.cz/ontology/ruian/Stat>
        )
    )
}
\end{lstlisting}
\end{figure}

\section{Discretization\label{discretization}}

The RDFRules implementation provides discretization functionality. The discretization tasks are, howerever, federated to the implemented discretization algorithms of the EasyMiner-Discretization library\footnote{\href{https://github.com/KIZI/EasyMiner-Discretization}{https://github.com/KIZI/EasyMiner-Discretization}}. The equifrequent and the equisize discretization were chosen for the purposes of this experiment. The count of intervals that is set to be created from the set of measured values while performing the equifrequent discretization determines how many a which rules are generated by the RDFRules algorithm. If the the values are discretized into a small number of intervals, more rules with should be geenerated but the measure values become coarse and information is lost. If creating too many intervals, more specific rules should be found but they happen to have lower support. When performing the equisize discretization, coarser rules are found for the intervals created for a higher support. 

To avoid guessing, which number of intervals and which minimal support suits best the preprocessed data, multiple discretizations were performed with different parameters for both discretization algorithms. The minimal support thresholds were calculated as absolute numbers of various percentages of observations in the discretized cubes. As it was already mentioned, the preprocessed cube has to be cut into subcubes with commensurable observations, measures in theses subcube have to be discretized separately and only after that the triples can be merged and performed the mining tasks on.

That means that the number of overall measurements multiplies by the number of distinct discretizations. A situation has to be avoided, when the instantiations of variable representing observations are involving observations not only from various subcubes. The approach to solve this problem no matter how many measures are assigned to each observation will be shown on a sample of data below.

\begin{figure}[h]
\begin{lstlisting}[language = turtle, caption={XYZ}, label={discsample},captionpos=b escapeinside={(*@}{@*)}]
@prefix qb: <http://purl.org/linked-data/cube#> .

<o1> qb:dataSet <original-dataset> ;
    <dimension1> <dimension1value1> ; <dimension2> <dimension2value1> ;
    <measure1> 25000 ;
    <measure2> 3 .

<o2> qb:dataSet <original-dataset> ;
    <dimension1> <dimension1value2> ; <dimension2> <dimension2value2> ;
    <measure1> 10000 ;
    <measure2> 10 .
 \end{lstlisting}
\end{figure}

Each application of a discretization algorithm will create a new measurement triple for each measure and observation with an object of the assigned interval based on the discretization algorithm and the parameter. The objects in triples assigning the observations to a data set will be changed to point to the particular subcube. In the example below two discretizations for each measure were performed on the two observations. In the example the same pair of discretizations were performed on the two distinct measures, but that does not have to be so. Assigning multiple triples of the same measure is fine as far as the measure is differently discretized. 

\begin{figure}[h]
\begin{lstlisting}[language = turtle, caption={XYZ}, label={discsample},captionpos=b escapeinside={(*@}{@*)}]
@prefix qb: <http://purl.org/linked-data/cube#> .
        
<o1> qb:dataSet <subcube1> ;
    <dimension1> <dimension1value1> ; <dimension2> <dimension2value1> ;
    <measure1> <subcube1_ef3_measure1_3>, <subcube1_es10_measure1_2> ;
    <measure2> <subcube1_ef3_measure2_2>, <subcube1_es10_measure2_1> .

<o2> qb:dataSet <subcube2> ;
    <dimension1> <dimension1value2> ; <dimension2> <dimension2value2> ;
    <measure1> <subcube2_ef3_measure1_3> , <subcube2_es10_measure1_2> ;
    <measure2> <subcube2_ef3_measure2_1> ,<subcube2_es10_measure2_1> .
\end{lstlisting}
\end{figure}

The discretized subcubes can be than merged into a singe data set a be performed mining tasks on. In each rule it has to be ensured, that the set of observations is limited to a certain subcube. For each cube in a rule the body of a rule has to contain an atom of a pattern \verb|(?o qb:dataSet AnyConstant)|. 

An alternative to creating subcubes based on the concept hierarchy in the master cube's dimensions is to use an unsupervised clustering algorithm (eg. k-means) to divide the observations into subcubes based on the proximity of their measures. After that the workflow is the same, the measures are discretized in the generated subcubes separately and than the subcubes are merged. But this brings two problems when used for the RDFRules algorithm.

\begin{enumerate}
    \item There is no clear interpretation of the generated subcubes, because their observation can belong to different levels of the concept hierarchy in a dimension. Unlike with the previous approach where a generated subcube could be described as for example \textit{Population in districts by age category}.
    \item For each distinct measure in the master cube a set of subcubes would have to be generated. So if the observations are assigned multiple measures (as the CZSO observations are) the number of observations would multiple with the number of measures and the observation URIs would have to be distinguished as one observation cannot be assigned to multiple data sets by the \verb|qb:dataSet| triple.
\end{enumerate}

% zkus si i tu equisize pro různé min supporty, jestli to nevygeneruje to same co ten ef

\section{Mining Tasks}

% 1 kostka + KG
% 2 kostky (nejlíp z jiného zdroje) + KG

% finding KG triples
% predicting measure intervals

\subsection{Relation between the pension expenses and the policital alignment of the state's government}

% TODO samotné bádání

\section{Discussion}