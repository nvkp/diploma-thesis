\chapter*{Conclusions}
\addcontentsline{toc}{chapter}{Conclusions}

The goal of this work was to explore the possibilities of enriching RDF data cube with the data from publicly available knowledge graphs and mining rules from this data. The section~\ref{combining} contains a description of steps needed in data pre-processing phase, a type of rules that can be mined from the data and how this type affects the interest measures of the RDFRules algorithm based on how the input data is structured. This knowledge was used when performing the experiment described in section \ref{experiment}, that yielded results confirming the validity of the suggested approach, which was also a goal of this work. The RDFRules framework proved to be well suited for this type of experiment, thanks to its native recognition of the $owl:sameAs$ predicates and ability to control the shaped of the generated rule with its powerful rule pattern syntax. The source codes of the mining procedures and sets of the found rules are available in a Github repository at \href{https://github.com/nvkp/diploma-thesis-code}{https://github.com/nvkp/diploma-thesis-code}.

The extraction and pre-processing of the data was very time-consuming, but there is a lot of room to automate not only this task. For the dimension values of the examined cubes, automatic \textit{record linkage} \cite{Sheth2013} tools could be used to automate the discovery of relevant triples in other data sources, which which the data set can be augmented. Also a tool can be imagined, that would examine the cubes' dimension value, which would infere the suitable way to cut the cube into the commensurable subcubes based on the hierarchy of the dimension value if they were described with SKOS or any other similar vocabulary, and on the characteristics of the measurements associated with the dimension values. Another tool could aid the post-processing phase by compensating for the interest measures' distortions based on the content of the rules and based on the structure of the cubes. 

Regarding the RDFRules framework core API, as described in \ref{note}, the frameworks permits a definition of a rule pattern matching rules, that the algorithm cannot generate, which results in zero rules found. A functionality to the framework could be added, that would check for the validity of a rule pattern, and it would either try to repair the rule pattern by rearranging the atom patterns in the rule pattern's body or it would simply provide a feedback to the user. Also the user-friendliness of both the core API and the Web UI would be improved if not only single alphabetical characters moreover, in alphabetical order from the head, are allowed for the names of variables in the rule patterns.

% taky že v rdfrules je možné vytvorit rule pattern, ktery neodpovída tomu, jak algoritmus generuje pravidla, takže by tam mohla být pridelaná funcionalita, která by ten pattern validovala, nebo se ho sama snazila opravit